\documentclass[12pt, oneside, a4paper]{article}
\usepackage{pscyr}
\usepackage[bookmarksopen]{hyperref}
\usepackage[warn]{mathtext}
\usepackage[english,russian]{babel}
\usepackage{amsmath,amssymb}
\usepackage{booktabs}
\usepackage{array}
\usepackage{float}
\usepackage{graphicx}
\usepackage{wrapfig}
\renewcommand{\rmdefault}{ftm}
\usepackage[utf8]{inputenc}
\DeclareSymbolFont{T2Aletters}{T2A}{cmr}{m}{it}
\everymath{\displaystyle}
\renewcommand{\baselinestretch}{1.65}
\begin{document}
\begin{titlepage}
\centerline{\large\bfseriesИНСТИТУТ ПРИКЛАДНОЙ ФИЗИКИ РАН}
\vspace{3.0cm}
\centerline{\large\bfseriesРЕФЕРАТ}
\vspace{0.2cm}
\centerline{\largeпо дисциплине <<Английский язык>>}
\vspace{0.3cm}
\centerline{на тему}
\vspace{0.2cm}
\begin{center}
\large
<<Процессы в ионосфере как прекурсоры землетрясений: краткий обзор экспериментальных наблюдений>>
\vspace{1.0cm}
\begin{flushright}
\normalsize
Выполнил: аспирант 1-го года\\очной формы обучения 
\end{flushright}
\normalsize
\begin{flushright}
\begin{tabular}{>{\flushright}p{3.0cm}>{\flushleft}p{7.0cm}}
Исполнитель: & Одзерихо Дмитрий Александрович\tabularnewline
Подпись: &\tabularnewline
%Научный руководитель: & Костров А.В., д.ф.-м.н.\tabularnewline\hline
%Подпись: &\tabularnewline\hline
Рецензент: & Филиппова Светлана Юрьевна \tabularnewline
Подпись: &\tabularnewline
\end{tabular}
\end{flushright}
\vfill
\vspace{0.7cm}
\centerline{Нижний Новгород}
\centerline{2011}
\end{center}
\end{titlepage}
\tableofcontents
\newpage
\section{Введение}
История изучения связи сейсмической активности с процессами в ионосфере прошла уже несколько этапов, начиная с энтузиазма, вызванного самим открытием, последовавшими вслед за ним публикациями, по большей части, спекулятивными, их жестокой критикой, и заканчивая систематическими исследованиями, которые привели к выработке  последовательной физической модели явления. 

Начало исследованиям сейсмо-ионосферной связи положило землетрясение 27 марта 1964 года в Аляске. Первые работы, описывающие возмущения в ионосфере как прекурсоры сейсмической активности, были посвящены колебаниям параметра \emph{foE} и модификации электронной плотности в ионосфере, предшествовавшим землетрясению в Ташкенте в 1966 году. Вслед за ними стали регулярно появляться работы, изучавшие эти феномены, проявлявшиеся в новых землетрясениях. 

Пионерские работы исследователей основывались на данных, полученных, в основном, от наземных ионостанций; однако, вскоре стали появляться работы, использовавшие результаты спутниковых измерений. 

Первые публикации, посвященные описываемой проблеме, характеризуются феноменологическим подходом к описанию явления, без привлечения строгого физического его обоснования. Тем не менее, развитие методик обработки информации в последующие годы позволило выработать надежную базу знаний.

Что касается физической интерпретации феномена, то существует две конкурирующие гипотезы образования сейсмо-ионосферной связи. Первая гипотеза описывает влияние акустической волны, генерируемой зоной землетрясения, на ионосферу; вторая, в качестве первопричины возмущения ионосферных параметров, рассматривает аномальное вертикальное электрическое поле, проникающее из области готовящегося землетрясения в ионосферу.  Изменения в проводимости воздуха можно рассматривать как дополнительный параметр в модели с электрическим полем. 

Изначально, главенствующей гипотезой была модель с акустической волной, возбуждаемой предсейсмической активностью истекающих газов~\cite{Hegai:1997}. Эта идея, тем не менее, еще не нашла своего строгого экспериментального подтверждения. Все известные зарегистрированные возмущения в ионосфере, вызванные колебаниями земной поверхности в результате даже сильных землетрясений, пренебрежимо малы.  Экспериментальные измерения возмущений в ионосфере, последовавших после землетрясения в Нортридже 17 января 1994 года, \emph{M}=6.7, показали, что амплитуда колебаний электронной плотности, связанных с акустической волной, генерируемой земной поверхностью, оказалась на 2--2.5 порядка ниже, чем фоновые колебания. Экспериментальные данные, подтверждаются теоретическими оценками, проведенными в работе~\cite{Davies_Archambeau:1998}, где наиболее аккуратно (на тот момент времени) решена задача возбуждения ионосферных возмущений акустическими волнами, возбуждаемых в ходе землетрясения. В качестве основного предположения авторы взяли тот факт, что землетрясение, само по себе, возбуждает колебания воздуха в гораздо большей степени, чем любой из его прекурсоров. Вычисления, выполненные в~\cite{Davies_Archambeau:1998}, показали, что относительное изменение электронной плотности в максимуме не превосходит $0.3\%$, что на два порядка меньше, чем естественная суточная вариация этого параметра, что делает эффект практически ненаблюдаемым. Таким образом, модель с акустической волной не позволяет объяснить наблюдаемое экспериментально сильное возмущение концентрации электронов непосредственно перед самим землетрясением. 

Все результаты и вычисления, упомянутые выше, позволяют утверждать, что модель акустической волны несостоятельна, поэтому дальнейший обзор будет посвящен обзору модели связи сейсмических и ионосферных явлений посредством электрического поля, генерируемого зоной землетрясения. Настоящая работа содержит в себе обсуждение экспериментальных наблюдений в околоземной плазме (акцент сделан на последние достижения) и их интерпретации с точки зрения выбранной гипотезы.

\section{Физическая модель сейсмо-ионосферной\\связи}
\subsection{Приповерхностные процессы}
Процессы, образующие физический механизм сейсмо-ионосферной связи, схематически изображены на~\mbox{Рис. \ref{fig:process-chain}}. 
\begin{figure}[H]
    \centering
    \includegraphics*[width=0.8\columnwidth]{process-chain}
    \caption{Диаграмма модели образования сейсмо-ионосферной связи.}
    \label{fig:process-chain}
\end{figure}
В области готовящегося землетрясения, помимо механического движения земной поверхности, наблюдаются геохимические явления, такие как испускание радона, инертных и парниковых газов. Подготовительным этапом для возбуждения ионосферных возмущений является формирование приповерхностной плазмы в виде разрозненных облаков, состоящих из составных ионов, образуемых в ходе ион-молекулярных реакций (после ионизации воздуха радоном) в приповерхностном слое атмосферы и присоединения к сформировавшимся ионам молекул воды. Большой дипольный момент молекул воды препятствует рекомбинации облаков составных ионов. В ходе электростатического взаимодействия разноименно заряженных ионных облаков в атмосфере образуются квазинейтральные области. В теории пылевой плазмы этот процесс называется коагуляцией. Коагуляция приводит к тому, что образовавшаяся в приповерхностной атмосфере область, обогащенная составными ионами, оказывается изолированной от ионосферы квазинейтральными облаками~\cite{Horanyi_Goertz:1990}.


Второй этап образования сейсмо-ионосферной связи --- это генерация аномального электрического поля. Известно, что непосредственно перед землетрясением наблюдаются интенсивные газовые выбросы из земной коры (в основном $CO_2$). Эти выбросы играют двоякую роль. Вызванные ими мощные потоки воздуха генерируют акустические волны и рассеивают квазинейтральную область из за слабого кулоновского взаимодействия в ней. В  течение короткого времени нижние слои атмосферы сильно ионизируются (оцениваемая концентрация $10^5-10^6$\,см$^{-3}$). Последующий процесс разделения зарядов ведет к возбуждению аномально сильного вертикального электрического поля. Одной из главных причин возникновения разделения зарядов является различие в подвижности разноименно-заряженных ионов, из которых состоит плазма нижних слоев атмосферы. Генерация аномального электрического поля --- финальная стадия в первом звене длинной цепи процессов, приводящих к образованию сейсмо-ионосферной связи. Стоит заметить, что при некоторых геофизических условиях (например, в условиях тумана), направление аномального электрического поля может совпадать с направлением естественного атмосферного электрического поля.

Прежде чем приступить к рассмотрению изменений в ионосфере, вызываемых аномальным электрическим полем,  подробнее обсудим  процессы, происходящие в приповерхностном слое атмосферы. Во-первых, выбросы подземных газов, в дополнение к их разрушающей нейтральную область роли, переносят аэрозоли с характерным размером пылевых частиц меньше микрона, что, как известно, усиливает электрическое поле из за уменьшения проводимости воздуха в их присутствии. Во-вторых, в сейсмически активных зонах регистрируется генерация электромагнитных волн низкочастотного диапазона: КНЧ, УНЧ и ОНЧ.  Технология их регистрации, идентификации и сепарации от низкочастотных волн грозового и техногенного происхождения на сегодняшний день развита в полной мере. Можно выделить два наиболее часто используемых подхода к решению указанной задачи: пространственная и поляризационная сепарация; однако, несмотря на успехи диагностики, физическое происхождение наблюдаемых электромагнитных волн попрежнему остается невыясненным~\cite{Jianguo:1989}.

Как было указано выше, приповерхностный слой атмосферы переходит, практически, в плазменное состояние, с концентрацией электронов, сравнимой с ионосферной.  Эта плазма оказывается под влиянием сильного электрического поля, что делает возможным генерацию потоков ускоренных частиц и развитие различного рода плазменных неустойчивостей. Оценим значение плазменной частоты для плазмы, состоящей из составных ионов $NO^-_3\cdot{}(H_2O)_n$, где n - число молекул воды, входящих в составной ион. В случае n=6, атомная масса составного иона равна $M=190$\,а.е., что эквивалентно $m=3.15\cdot{}10^{-22}$\,г, и, приняв концентрацию заряженных частиц в облаке порядка $10^6$\,см$^{-3}$, получаем, что значение частоты $f_p=\omega_p/2\pi\sim{}16.9$\,кГц лежит внутри ОНЧ диапазона. Учтем, что как концентрация плазмы, так и масса составных ионов может изменяться в широких пределах; в этом случае значение частоты $f_p$ перекрывает весь диапазон УНЧ-КНЧ-ОНЧ. Тепловой шум вблизи плазменной частоты может быть значительно усилен электрическим полем; таким образом, можно ожидать развития неустойчивостей тормозного и черенковского типов и, соответственно, излучения электромагнитных волн указанных диапазонов непосредственно из пылевой плазмы; возможна генерация ионных и пылевых акустических волн~\cite{Kikuchi:2001}. 
\subsection{Процессы в ионосфере}
Аномальное электрическое поле проникает внутрь $Е-$слоя ионосферы и создает возмущает ее параметры, что было экспериментально зарегистрировано. В зависимости от направления электрического поля (вверх или вниз), возможно уменьшение или увеличение электронной плотности. Форма области, в которой генерируется электрическое поле (округлая или вытянутая вдоль некоторого направления) определяет форму возмущенной области ионосферы. В любом случае,  возмущающей ионосферу силой является перпендикулярная геомагнитным линиям компонента аномального электрического поля. В ситуации, когда аномальное электрическое поле направлено вниз к земной поверхности, в ионосфере над областью готовящегося землетрясения формируется спорадический $E-$слой, что было зафиксировано в экспериментах и теоретически обосновано.

В силу эквипотенциальности геомагнитных линий, электрического поле проникает в верхние слои ионосферы практически без потерь. В $F-$слое ионосферы существенными представляются два эффекта. Во-первых, джоулев нагрев в области максимально электрической проводимости  приводит к генерации акустических волн, что, в свою очередь, способно возбудить мелкомасштабные колебания электронной плотности в ионосфере. Эти процессы наблюдаются на различных ионосферных высотах при помощи радио и оптического мониторинга. Второй, наиболее изученный эффект, заключается в формировании крупномасштабных возмущений концентрации электронов в $F-$слое ионосферы. Этот эффект был зарегистрирован как спутниковыми методами диагностики, так и комбинированными, с использованием наземных ионостанций и сети GPS. В силу сложного характера движения электрона в скрещенных электрическом и геомагнитных полях, крупномасштабные возмущения плотности электронов в $F-$слое могут наблюдаться  не только над  эпицентром землетрясения, но также на более низких широтах~\cite{Lui:2004}. 
\subsection{Процессы в магнитосфере}
В верхних слоях ионосферы возможно развитие следующих эффектов. Мелкомасштабные возмущения ионосферы, проникая в магнитосферу вдоль геомагнитных линий, создают ориентированные вдоль поля области пониженной плотности, на которых возможно рассеяние ОНЧ волн грозового или техногенного происхождения; это приводит к увеличению амплитуды ОНЧ волн внутри магнитных трубок, проходящих вдоль областей генерации аномального электрического поля. Из за дрейфового движения плазмы форма возмущенных областей в магнитосфере будет отличаться от таковой вблизи земной поверхности, а именно, вытянется вдоль зонального направления. Циклотронная неустойчивость ОНЧ волн приводит к стимулированному высыпанию  энергичных частиц из радиационных поясов Земли~\cite{Pulinets_Hegai:2002a}.
\subsection{Процессы в $D-$слое ионосферы}
Сложная цепочка процессов в атмосфере, ионосфере и магнитосфере приводит к стимулированному высыпанию частиц, ионизирующему нижние слои ионосферы. Эта ионизация ведет к увеличению электронной плотности в $D-$слое ионосферы, что эквивалентно её понижению; это изменяет условия прохождения электромагнитных волн в широком диапазоне от ОНЧ до УКВ. Аномальное изменение прохождения радиоволн, предшествовавшее сильным землетрясениям, было экспериментально зарегистрировано~\cite{Gufeld:1992}.  
\section{Иерархия электромагнитных прекурсоров\\землетрясений}
Вышеприведенный  обзор позволяет выстроить все электромагнитные явления в ионосфере, причиной которых является предсейсмическая активность, в иерархическом порядке~\cite{Krider_Roble:1986_2002}. 

Так как региструемый уровень УНЧ-ОНЧ шумов являтся результатом развития различных неустойчивостей в атмосферной плазме, то их интенсивность должна быть свзяана с выбросами радона; существующие записи об уровнях сейсмо-ионосферной активности и сожержании радона в соотвествующих областях потдвержадют эту гипотезу. Таким образом, геохимические процессы в приповерхностном слое атмосферы должны рассматриваться как первопричина образования сейсмо-ионосферной связи. 

Следующим эффектом в иерархии процессов, вовлеченных в сейсмо-ионосферную связь, является  генерация аномального электрического поля. Он проявляется в финальной стадии подготовки землетрясения, т.е. после того, как произошли все необходимые изменения в атмосферной плазме. Важно, чтобы напряженность аномального электрического поля была не меньше нескольких киловольт на метр, т.е. быть значительно выше напряженности естественного электрического поля при нормальных погодных условиях.

За эффектом генерации аномального электрического поля следуют эффекты излучения электромагнитных волн в диапазоне ОНЧ-ВЧ-СВЧ, что является результатом физических процессов, происходящих в аэрозольной плазме в присутствии сильного электрического поля.

Последними в описываемой иерархии стоят всевозможные возмущения ионосферы, включая мелко- и крупномасштабные колебания электронной плотности, излучение в видимом диапазоне длин волн, а также, такие эффекты в плазмасфере и магнитосфере, как стимулированное высыпание частиц и аномальное ОНЧ излучение.

Аномалии в прохождении ОНЧ сигналов, в свою очередь, являются следствием модификации $D-$слоя ионосферы в результате высыпаний энергичных частиц из земной магнитосферы.
\section{Заключение}
Настоящая работа содержит описание последних достижений в понимании физического механизма образования связи между сейсмическими и ионосферными процессами. Результаты этих исследований могут быть использованы при разработке технических решений краткосрочного предсказания землетрясений, основанных на регистрации их прекурсоров в ионосфере. 
\newpage
\bibliographystyle{unsrt}
\bibliography{bib}

\end{document}
