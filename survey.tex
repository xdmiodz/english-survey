\documentclass[12pt, oneside, a4paper]{article}
\usepackage{pscyr}
\usepackage[bookmarksopen]{hyperref}
\usepackage[warn]{mathtext}
\usepackage[english,russian]{babel}
\usepackage{amsmath,amssymb}
\usepackage{booktabs}
\usepackage{array}
\usepackage{graphicx}
\usepackage{wrapfig}
\renewcommand{\rmdefault}{ftm}
\usepackage[utf8]{inputenc}
\DeclareSymbolFont{T2Aletters}{T2A}{cmr}{m}{it}
\everymath{\displaystyle}
\usepackage{float}
\renewcommand{\baselinestretch}{1.65}
\begin{document}
%\begin{titlepage}
%\centerline{\large\bfseriesИНСТИТУТ ПРИКЛАДНОЙ ФИЗИКИ РАН}
%\vspace{3.0cm}
%\centerline{\large\bfseriesРЕФЕРАТ}
%\vspace{0.2cm}
%\centerline{\largeпо дисциплине <<Англииский язык>>}
%\vspace{0.3cm}
%\centerline{на тему}
%\vspace{0.2cm}
%\begin{center}
%\large
%<<История экспериментальных подтверждений электромагнитной теории света Дж.~Мак%свелла>>
%\vspace{1.0cm}
%\begin{flushright}
%\normalsize
%Выполнил: аспирант 1-го года\\очной формы обучения 
%\end{flushright}
%\normalsize
%\begin{flushright}
%\begin{tabular}{|>{\flushright}p{3.0cm}|>{\flushleft}p{7.0cm}|}
%\hline
%Исполнитель: & Одзерихо Дмитрий Александрович\tabularnewline\hline
%Подпись: &\tabularnewline\hline
%Научный руководитель: & Костров А.В., д.ф.-м.н.\tabularnewline\hline
%Подпись: &\tabularnewline\hline
%Рецензент: & Булюбаш Б.В., к.ф.-м.н., доц. НГТУ\tabularnewline\hline
%\end{tabular}
%\end{flushright}
%\vfill
%\centerline{Нижний Новгород}
%\centerline{2011}
%\end{center}
%\end{titlepage}
%\tableofcontents
%\newpage
\section{Введение}
История изучения взаимосвязи сейсмической активности с процессами в ионосфере прошла уже несколько этапов, начиная с энтузиазма, вызванного самим открытием и последовавшими вслед за этим публикациями, по-большей части, спекулятивными и их жестокой критикой, и заканчивая систематическими исследованяими, которые привели к выработке  последовательной физической модели явления. Общепринятым считается, что землетрясения на Аляске 27 марта 1964 года дало толчок исследованию связи между сейсмической и ионосферной активностями. Среди всех работ того времени, посвященных изучению электромагнитных и ионосферных явлений, связанных тем или иным образом, с землетрясением, можно найти, по-крайней мере две, в которых упоминяются явления, непосредственно ему \emph{предшествовавшие}~\cite{Moore:1964,Davies_Baker:1965}. Первыми публикациями, описывающие изменение ионосферных параметров как прекурсоры сейсмической активности была работа \cite{Antselevich:1971}, посвященная вариации параметра \emph{foE}, предшествовашего землетрясение в Ташкенте в 1966 году, и работа~\cite{Datchenko:1972}, изучавшая изменение электронной плотности в ионосфере перед тем же самым ташкентским землетрясением. Вслед за ними стали регулярно появляться работы, посвященные изучению новых землетрясений. Описанные выше работы были основаны на измерениях параметров ионосферы, полученных, в основном, от наземных ионостанций; однако, вскоре стали появляться работы, использовавшие данные с исскуственных спутников Земли~\cite{Gokhberg:1983}. Самые первые работы, посвященные описываемой проблеме, характеризуются феноменологическим подходом к описанию явления, без привлечения строгого физического его обоснования. Тем не менее, развитие методик обработки инофрмации в последующие годы позволило накопить надежную базу знаний. Подробное описание долгой истории изучения сейсмо-ионосферной связи  можно найти в обзорах \cite{Liperovsky:1990,Gaivoronskaya:1991} и других.

Что касается физической интерпетации, существовало две конкурирущие гипотезы, описывающие явление. Первая гипотеза описывала влияние акустической волны, генерируемой зоной землетрясения, на ионосферу; вторая в качесвте первопричины модификации ионсферных параметров рассматривала аномальное вертикальное электрическое поле, проникающее из области землетрясения в ионосферу.  Мы можем рассматривать изменения в проводимости воздуха как дополнительный параметр в модели с электрическим полем. Изначально, акустическая гипотеза была главенствующей. В  работе ~\cite{Mareev:2002} описывает идею возбуждения акустической волны предсейсмчисечекой активностью истекающих газов. Эта идея, тем не менее, еще не нашла своего строгого эксперементального подтверждения. До сих пор, все зарегестрированные возмущений в ионосфере, вызванные колебаниями земной поверхности в результате даже сильных землетрясений, были пренебрежимо малы. Авторы работы~\cite{Calais_Minster:1995}, использовавшие технологию GPS TEC для экспериментального измерения возмущений в ионосфере, последовавших после землетрясения в Нортридже 17 января 1994 года, \emph{M}=6.7, отмечают, что вариация электронной плотности, вызванная с акустическими колебаниями, генеририемыми земной поверхностью,оказалась на 2--2.5 порядка ниже, чем фоновые колебания. Экспериментальные данные, полученные в этой работе, подтверждаются теоритическими оценками, проведенные в~\cite{Davies_Archambeau:1998}, в которой наиболее аккуратно (на тот момент времени) решена задача возбуждения ионосференых возмущений акустическими волнами, генерируемыми в ходе землетрясения. В качестве основного предположения авторы взяли тот факт, что землетрясение, само по себе, возбуждает колебания воздуха в гораздо большей степени, чем любой из его прекурсоров. Вычисления, выполненные авторами, показали, что относительное изменение электронной плотности в максимуме не превосходит $0.3\%$, что на два порядка меньше, чем суточная вариация этого параметра, что делает эффект практически ненаблюдаемым. Очевидно, что гипотеза с акустической волной не позволяет объяснить наблюдаемое экспериментально сильное возмущение концентрации электронов непосредственно перед самим землетрясением~\cite{Lui:2004}. Все результаты и вычисления, упомянутые выше, позволяют утверждать, что гипотеза с акустической волной несостоятельна. Поэтому дальнейший обзор будет посвящен обзору гипотезы с электрическим полем. Настоящая работа содержит в себе обсуждение экспериментальных наблюдений в околоземной плазме (с упором на последние достижения) и их интепретации с точки зрения выбранной модели.

\section{Физическая модель сейсмо-ионосферной связи}
\subsection{Приповерхностные процессы}
Процессы, вовлеченные в физический механизм сейсмо-ионосферной связи, схематически изображены на~\mbox{Рис. \ref{fig:process-chain}}. В области  предстоящего землетрясения (размеры которой определяются силой будущего землетрясения~\cite{Pulinets:2004a}), помимо чисто механических процессов, наблюдаются геохимичеческие явления, такие как испускание радона, инертных и парниковых газов. Подготовительным этапом для возбуждения ионосферных возмущений является формирование приповерхностной плазмы в виде разрозненных ионных облаков, образуемых в ходе ион-молекулярных реакций (после ионизации радоном) в приповерхностном слое атмосферы и присоединенных к сформировавшимся оинома молекул воды. Большой дипольный момент молекул воды препятсвует рекомбинации ионных облаков. В работе~\cite{Pulinets_Boyarchuk:2004} детально описан механизм формирования приповерзностных ионных облаков. В ходе электростатического взаимодействия разноименно-заряженных ионных облаков образуются квази-нейтральные области. В теории пылевой плазмы этот процесс называется коагуляцией~\cite{Horanyi_Goertz:1990}. После этого в приповерхностой атмосфере образуются области, богатые ионами, скрытые за сформировавшимися нейтральными облаками~\cite{Pulinets:2002a}.
\begin{figure}[H]
    \centering
    \includegraphics*[width=0.8\columnwidth]{process-chain}
    \caption{Блочная диаграмма модели сейсмо-ионосферной связи.}
    \label{fig:process-chain}
\end{figure}

Второй этап --- это генерация аномального электрического поля. Известно~\cite{Voitov_Dobrovolsky:1994}, что непосредственно перед землетрясением в земной коре наблюдаются интенсивные газовые выбросы (в основном $CO_2$). Эти выбросы играют двоякую роль. Мощные потоки воздуха, вызванные выбросами газов, генерирут акустические волны и уничтожают квази-нейтральные области из за слабого кулуновского взаимодействия в них. В результате, в течение короткого времени, приповерхностный слой атмосферы сильно ионизируется (оцениваемая концентрация $10^5--10^6$\,см$-3$). Последующий процесс разделения зарядов ведет к возбуждению аномально сильного вертикального электрического поля. Одной из главных причин возникновения разделения зарядов является различие в подвижности разноименно-заряженных ионов, из которых состоит атмосферная плазма. Генерация аномального электрического поля --- финальная стадия в первом звене длинной цепи процесов, приводящих к сейсмо-ионосферному взаимодействию. Стоит заметить, что при некоторых геофизических условиях (нпример, в условиях тумана), направление аномального электрического поля может совпадать с направлением естественного атмосферного электрического поля. Документально зарегестрированны случаи генерации сейсмического электричества как прекурсора сильного землетрясения~\cite{Jianguo:1989,Vershinin:1999}.

Перед тем, как приступить к рассмотрению изменений в ионосфере, вызванных аномальным электрическим полем, обсудим подбробнее процессы, происходящие в приповерхностном слое атмосферы. Во-первых, выбросы подземных газов, в дополнение к их разрушающей нейтральные кластеры роли, могут переносить аэрозоли, с характерным размером пылевых частиц меньше микрона, которые, как известно, способны усилить электрическое поле из за уменьшения проводимости воздуха в их присутствии~\cite{Krider_Roble:1986}. Расчет увеличения электрического поля в воздухе в присутствии аэрозолей можно найти в~\cite{Pulinets:2000}. Во-вторых, в сейсмически активных зонах регистрировалиась генерация электромагнитных волн низкочастоного диапазона: КНЧ, УНЧ и ОНЧ~\cite{Nagao:2002}.  Технология их регистрации, идентификации и сепарации от низкочастонтых волн грозового и техногенного происхождения на сегодняшний день достаточна развита. По-крайней мере, используется два подхода к решению указанной задачи: пространственная~\cite{Ismaguilov:2001} и поляризационная~\cite{Hattori:2002} сепарация; однако, физическое происхождение наблюдаемых электромагнитных волн по-прежнему остается невыясненным. Как было указано выше, приповерхностный слой атмосферы переходит, практически, в плазменное состояние, с концентрацией, сравнимой с ионосферной.  К тому же, эта плазма оказывается под влиянием сильного электрического поля, в которой возможна генерция потоков ускоренных частиц и образование различного рода плазменных нестабильностей. Оценка для величины плазменной частоты для плазмы, состоящей из составных ионов $NO^-_3\cdot{}(H_2O)_n$, где n - число молекул воды, входящих в составной ион. В случае n=6, атомная масса иона равна $M=190$\,а.е., что экваивалентно $m=3.15\cdot{}10^{-22}$, и, приняв концентрацию заряженных частиц в облаке порядка $10^6$\,см$^{-3}$, получаем, что значение частоты $f_p=\omega_p/2\pi\sim{}16.9$\,кГц лежит внутри ОНЧ диапазона. Учтем, что как концентрация плазмы, так и масса составных ионов может изменяться в широких пределах; в этом случае значение частоты $f_p$ покрывает весь диапазон УНЧ-КНЧ-ОНЧ. Тепловой шум вблизи плазменной частоты может быть значительно усилен процессами генерации электрического поля и ускорения частиц; таким образом, можно ожидать развития неустойчивостей тормозного и черенковского типов; возможна генерация ионных и пылевах акуситих волн. Подробное описание неусточивостей, способных развиться в пылевой плазме, можно найти в~\cite{Kikuchi:2001}. 
\subsection{Эффекты в ионосфере, вызываемые аномальным электрическим полем}
Аномальное элекктрическое поле проникает внутрь $Е-$слоя ионосферы и создает возмущения ее параметров, что было экспериментально зарегистрировано ~\cite{Liperovsky:2000}. В зависимости от направления электричесого поля (вверх или вниз), возможно уменьшение или увеличение электронной плотности~\cite{Pulinets:1998}. Форма области, в которой генерируется электрическое поле (округлая иили вытянутаявдоль некоторого направления) определяет форму возмущенной области ионосферы. В любом случае,   возмущающей ионосферу силой является перпендикулярная геомагнитным линиям компонента аномального электрическорго поля. В ситуации, когда аномальное электрическое поле направлено вниз к земной поверхности, в ионосфере над областью готовящегося землетрясения формируется спорадический $E-$слой, что было зафиксировано в экспериментах~\cite{Ondoh_Hayakawa:1999} и теоритечески обосновано~\cite{Kim:1994}.

В силу эквипотенциальности геомагнитных линий, электрического поле проникает в верхние слои ионосферы практически без потерь. В $F-$слое ионосферы существенными представляются два эффекта. Во-первых, жжоулев нагрев в области максимально электрической проводимости  приводит к генерации акустических волн, что, в свою очередб, способно возбудить мелкомасштабную неустойчивость электронной плотности в ионосфере~\cite{Hegai:1997}. Эти процессы проявляются в виде периодических колебаниях электронной плотности, наблюдаемых на различных ионосферных высотах при помощи радио и оптического мониторинга~\cite{Chmyrev:1997}. Второй, наиболее изученный эффект, заключается в формировании крупномасштабных возмущений концентрации элкектронов в $F2-$слое ионосферы~\cite{Pulinets_Legenka:2003}. Этот эффект был зарегестрирован как спутниковыми методами дагностики, так и комбинированними, с использованием наземных ионостанций и сети GPS~\cite{Lui:2004}. В силу сложного характера движения электрона в скрещенных электричском и геомагнитных полях, крупномасштабные возмущения плотности электронов в $F-$слое могут наблюдаться  не только над  эпицентром землетрясения, но также на более низких широтах. 
\subsection{Эффекты в магнитосфере}
В верхних слоях ионосферы возможно развитие следующих эффектов. Мелкомасштабные возмущения, проникая в магнитосферу вдоль геомагнитных линий, создают ориентированные вдоль поля области пониженной плотности, на которых возможно рассеяние ОНЧ волн грозового или техногенного происхождения~\cite{Kim_Hegai:1997}; это приводит к увеличению амлитуды ОНЧ волн внутри магнитных трубок, проходящих вдоль областей генерации аномального электрического поля~\cite{Shklyar_Nagano:1998}. Из за дрейфого движения плазмы форма возмущенных областей в магнитосфере будет отличаться от таковой вблизи земной поверхности, а именно, вытянется вдоль зонального направления~\cite{Kim_Hegai:1997}. Циклотронная неустойчиовость ОНЧ волн приводит к стимулированному высыпанию  энергичных частиц из радиационных поясов Земли~\cite{Galper:1995}.
\subsection{Эффекты в $D-$слое ионосферы}
Сложная цепочка процессов в атмосфере, ионоисфере и магнитосфере приводит к стимулированному высыпанию частиц, ионизурующему нижние слои ионосферы. Эта ионизация ведет к увеличению электрооной плотности в $D-$слое ионосферы, что эквивалентно её понижению~\cite{Kim:2002}, что изменяет условия прохождения электромагнитных волн в широком диапазоне от ОНЧ до УКВ. Аномальное изменение прохождения радиоволн, предшествовавшее сильным землетрясениям, было экспериментально зарегистрировано~\cite{Gufeld:1992}.  
\section{Заключение}
Настоящая работа содержит описание последних достижений в понимании физического механизма образования связи между сейсмическими и ионосферными процессами. Результаты этих исследований могут быть использованы при разработке технических решений раннего предсказания землетрясений, основанных на регистрации их прекурсоров в ионосфере. 
\newpage
\bibliographystyle{plain}
\bibliography{bib}

\end{document}
